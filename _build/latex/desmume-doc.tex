%% Generated by Sphinx.
\def\sphinxdocclass{report}
\documentclass[letterpaper,10pt,english]{sphinxmanual}
\ifdefined\pdfpxdimen
   \let\sphinxpxdimen\pdfpxdimen\else\newdimen\sphinxpxdimen
\fi \sphinxpxdimen=.75bp\relax
\ifdefined\pdfimageresolution
    \pdfimageresolution= \numexpr \dimexpr1in\relax/\sphinxpxdimen\relax
\fi
%% let collapsable pdf bookmarks panel have high depth per default
\PassOptionsToPackage{bookmarksdepth=5}{hyperref}

\PassOptionsToPackage{warn}{textcomp}
\usepackage[utf8]{inputenc}
\ifdefined\DeclareUnicodeCharacter
% support both utf8 and utf8x syntaxes
  \ifdefined\DeclareUnicodeCharacterAsOptional
    \def\sphinxDUC#1{\DeclareUnicodeCharacter{"#1}}
  \else
    \let\sphinxDUC\DeclareUnicodeCharacter
  \fi
  \sphinxDUC{00A0}{\nobreakspace}
  \sphinxDUC{2500}{\sphinxunichar{2500}}
  \sphinxDUC{2502}{\sphinxunichar{2502}}
  \sphinxDUC{2514}{\sphinxunichar{2514}}
  \sphinxDUC{251C}{\sphinxunichar{251C}}
  \sphinxDUC{2572}{\textbackslash}
\fi
\usepackage{cmap}
\usepackage[T1]{fontenc}
\usepackage{amsmath,amssymb,amstext}
\usepackage{babel}



\usepackage{tgtermes}
\usepackage{tgheros}
\renewcommand{\ttdefault}{txtt}



\usepackage[Bjarne]{fncychap}
\usepackage{sphinx}

\fvset{fontsize=auto}
\usepackage{geometry}


% Include hyperref last.
\usepackage{hyperref}
% Fix anchor placement for figures with captions.
\usepackage{hypcap}% it must be loaded after hyperref.
% Set up styles of URL: it should be placed after hyperref.
\urlstyle{same}


\usepackage{sphinxmessages}




\title{desmume\sphinxhyphen{}doc}
\date{May 29, 2021}
\release{}
\author{Jhynjhiruu}
\newcommand{\sphinxlogo}{\vbox{}}
\renewcommand{\releasename}{}
\makeindex
\begin{document}

\pagestyle{empty}
\sphinxmaketitle
\pagestyle{plain}
\sphinxtableofcontents
\pagestyle{normal}
\phantomsection\label{\detokenize{index::doc}}


\sphinxAtStartPar
desmume\sphinxhyphen{}doc aims to document the DeSmuME emulator’s Lua interface, because it is poorly documented and this makes it a nightmare to work with.
The interface is powerful and allows quite a lot of useful scripting, but there doesn’t appear to be a single listing of all of the features.

\sphinxAtStartPar
Please note: desmume\sphinxhyphen{}doc does not cover the Lua language itself, only DeSmuME’s Lua interface, so this is not an introduction to Lua programming.


\chapter{Module index}
\label{\detokenize{modindex:module-index}}\label{\detokenize{modindex::doc}}\begin{itemize}
\item {} 
\sphinxAtStartPar
{\hyperref[\detokenize{mods:emu}]{\sphinxcrossref{\DUrole{std,std-ref}{emu}}}}

\item {} 
\sphinxAtStartPar
{\hyperref[\detokenize{mods:gui}]{\sphinxcrossref{\DUrole{std,std-ref}{gui}}}}

\item {} 
\sphinxAtStartPar
{\hyperref[\detokenize{mods:stylus}]{\sphinxcrossref{\DUrole{std,std-ref}{emu}}}}

\item {} 
\sphinxAtStartPar
{\hyperref[\detokenize{mods:savestate}]{\sphinxcrossref{\DUrole{std,std-ref}{savestate}}}}

\item {} 
\sphinxAtStartPar
{\hyperref[\detokenize{mods:memory}]{\sphinxcrossref{\DUrole{std,std-ref}{memory}}}}

\item {} 
\sphinxAtStartPar
{\hyperref[\detokenize{mods:joypad}]{\sphinxcrossref{\DUrole{std,std-ref}{joypad}}}}

\item {} 
\sphinxAtStartPar
{\hyperref[\detokenize{mods:input}]{\sphinxcrossref{\DUrole{std,std-ref}{input}}}}

\item {} 
\sphinxAtStartPar
{\hyperref[\detokenize{mods:movie}]{\sphinxcrossref{\DUrole{std,std-ref}{movie}}}}

\item {} 
\sphinxAtStartPar
{\hyperref[\detokenize{mods:sound}]{\sphinxcrossref{\DUrole{std,std-ref}{sound}}}}

\item {} 
\sphinxAtStartPar
{\hyperref[\detokenize{mods:bit}]{\sphinxcrossref{\DUrole{std,std-ref}{bit}}}}

\item {} 
\sphinxAtStartPar
{\hyperref[\detokenize{mods:agg}]{\sphinxcrossref{\DUrole{std,std-ref}{agg}}}}

\item {} 
\sphinxAtStartPar
{\hyperref[\detokenize{mods:controller}]{\sphinxcrossref{\DUrole{std,std-ref}{controller}}}}

\end{itemize}


\section{Modules}
\label{\detokenize{mods:modules}}\label{\detokenize{mods::doc}}

\subsection{emu}
\label{\detokenize{mods:emu}}\label{\detokenize{mods:id1}}
\sphinxAtStartPar
The \sphinxcode{\sphinxupquote{emu}} module contains general emulator functions.
\begin{itemize}
\item {} 
\sphinxAtStartPar
{\hyperref[\detokenize{mods:frameadvance}]{\sphinxcrossref{\DUrole{std,std-ref}{frameadvance}}}}

\item {} 
\sphinxAtStartPar
{\hyperref[\detokenize{mods:gamecode}]{\sphinxcrossref{\DUrole{std,std-ref}{gamecode}}}}

\item {} 
\sphinxAtStartPar
{\hyperref[\detokenize{mods:smallgamecode}]{\sphinxcrossref{\DUrole{std,std-ref}{smallgamecode}}}}

\item {} 
\sphinxAtStartPar
{\hyperref[\detokenize{mods:speedmode}]{\sphinxcrossref{\DUrole{std,std-ref}{speedmode}}}}

\item {} 
\sphinxAtStartPar
{\hyperref[\detokenize{mods:wait}]{\sphinxcrossref{\DUrole{std,std-ref}{wait}}}}

\item {} 
\sphinxAtStartPar
{\hyperref[\detokenize{mods:pause}]{\sphinxcrossref{\DUrole{std,std-ref}{pause}}}}

\item {} 
\sphinxAtStartPar
{\hyperref[\detokenize{mods:pause}]{\sphinxcrossref{\DUrole{std,std-ref}{unpause}}}}

\item {} 
\sphinxAtStartPar
{\hyperref[\detokenize{mods:emulateframe}]{\sphinxcrossref{\DUrole{std,std-ref}{emulateframe}}}}

\item {} 
\sphinxAtStartPar
{\hyperref[\detokenize{mods:emulateframefastnoskipping}]{\sphinxcrossref{\DUrole{std,std-ref}{emulateframefastnoskipping}}}}

\item {} 
\sphinxAtStartPar
{\hyperref[\detokenize{mods:emulateframefast}]{\sphinxcrossref{\DUrole{std,std-ref}{emulateframefast}}}}

\item {} 
\sphinxAtStartPar
{\hyperref[\detokenize{mods:emulateframeinvisible}]{\sphinxcrossref{\DUrole{std,std-ref}{emulateframeinvisible}}}}

\item {} 
\sphinxAtStartPar
{\hyperref[\detokenize{mods:redraw}]{\sphinxcrossref{\DUrole{std,std-ref}{redraw}}}}

\item {} 
\sphinxAtStartPar
{\hyperref[\detokenize{mods:getframecount}]{\sphinxcrossref{\DUrole{std,std-ref}{getframecount}}}}

\item {} 
\sphinxAtStartPar
{\hyperref[\detokenize{mods:getlagcount}]{\sphinxcrossref{\DUrole{std,std-ref}{getlagcount}}}}

\item {} 
\sphinxAtStartPar
{\hyperref[\detokenize{mods:lagged}]{\sphinxcrossref{\DUrole{std,std-ref}{lagged}}}}

\item {} 
\sphinxAtStartPar
{\hyperref[\detokenize{mods:emulating}]{\sphinxcrossref{\DUrole{std,std-ref}{emulating}}}}

\item {} 
\sphinxAtStartPar
{\hyperref[\detokenize{mods:atframeboundary}]{\sphinxcrossref{\DUrole{std,std-ref}{atframeboundary}}}}

\item {} 
\sphinxAtStartPar
{\hyperref[\detokenize{mods:registerbefore}]{\sphinxcrossref{\DUrole{std,std-ref}{registerbefore}}}}

\item {} 
\sphinxAtStartPar
{\hyperref[\detokenize{mods:registerafter}]{\sphinxcrossref{\DUrole{std,std-ref}{registerafter}}}}

\item {} 
\sphinxAtStartPar
{\hyperref[\detokenize{mods:registerstart}]{\sphinxcrossref{\DUrole{std,std-ref}{registerstart}}}}

\item {} 
\sphinxAtStartPar
{\hyperref[\detokenize{mods:registerexit}]{\sphinxcrossref{\DUrole{std,std-ref}{registerexit}}}}

\item {} 
\sphinxAtStartPar
{\hyperref[\detokenize{mods:persistglobalvariables}]{\sphinxcrossref{\DUrole{std,std-ref}{persistglobalvariables}}}}

\item {} 
\sphinxAtStartPar
{\hyperref[\detokenize{mods:message}]{\sphinxcrossref{\DUrole{std,std-ref}{message}}}}

\item {} 
\sphinxAtStartPar
{\hyperref[\detokenize{mods:print}]{\sphinxcrossref{\DUrole{std,std-ref}{print}}}}

\item {} 
\sphinxAtStartPar
{\hyperref[\detokenize{mods:openscript}]{\sphinxcrossref{\DUrole{std,std-ref}{openscript}}}}

\item {} 
\sphinxAtStartPar
{\hyperref[\detokenize{mods:reset}]{\sphinxcrossref{\DUrole{std,std-ref}{reset}}}}

\item {} 
\sphinxAtStartPar
{\hyperref[\detokenize{mods:addmenu}]{\sphinxcrossref{\DUrole{std,std-ref}{addmenu}}}}

\item {} 
\sphinxAtStartPar
{\hyperref[\detokenize{mods:setmenuiteminfo}]{\sphinxcrossref{\DUrole{std,std-ref}{setmenuiteminfo}}}}

\item {} 
\sphinxAtStartPar
{\hyperref[\detokenize{mods:registermenustart}]{\sphinxcrossref{\DUrole{std,std-ref}{registermenustart}}}}

\item {} 
\sphinxAtStartPar
{\hyperref[\detokenize{mods:register3devent}]{\sphinxcrossref{\DUrole{std,std-ref}{register3devent}}}}

\item {} 
\sphinxAtStartPar
{\hyperref[\detokenize{mods:set3dtransform}]{\sphinxcrossref{\DUrole{std,std-ref}{set3dtransform}}}}

\end{itemize}


\subsubsection{\sphinxstyleliteralintitle{\sphinxupquote{emu.frameadvance()}}}
\label{\detokenize{mods:emu-frameadvance}}\label{\detokenize{mods:frameadvance}}
\sphinxAtStartPar
Steps emulation one frame. Can be used when overriding the main emulation loop.


\subsubsection{\sphinxstyleliteralintitle{\sphinxupquote{emu.gamecode()}}}
\label{\detokenize{mods:emu-gamecode}}\label{\detokenize{mods:gamecode}}
\sphinxAtStartPar
Retrieves the full game code of the currently loaded ROM as a string.


\subsubsection{\sphinxstyleliteralintitle{\sphinxupquote{emu.smallgamecode()}}}
\label{\detokenize{mods:emu-smallgamecode}}\label{\detokenize{mods:smallgamecode}}
\sphinxAtStartPar
Retrieves the game code excluding the region code of the currently loaded ROM as a string.


\subsubsection{\sphinxstyleliteralintitle{\sphinxupquote{emu.speedmode(mode)}}}
\label{\detokenize{mods:emu-speedmode-mode}}\label{\detokenize{mods:speedmode}}
\sphinxAtStartPar
\sphinxcode{\sphinxupquote{mode}} is a string with a value of any of \{\sphinxcode{\sphinxupquote{normal}}, \sphinxcode{\sphinxupquote{nothrottle}}, \sphinxcode{\sphinxupquote{turbo}}, \sphinxcode{\sphinxupquote{maximum}}\}.

\sphinxAtStartPar
Sets the emulation speed when using \sphinxcode{\sphinxupquote{emu.frameadvance}}.


\subsubsection{\sphinxstyleliteralintitle{\sphinxupquote{emu.wait()}}}
\label{\detokenize{mods:emu-wait}}\label{\detokenize{mods:wait}}
\sphinxAtStartPar
Tells the emulator to wait while the script does calculations. Note that hotkeys are not disabled, so e.g. a savestate could still be loaded while the emulator is waiting.


\subsubsection{\sphinxstyleliteralintitle{\sphinxupquote{emu.pause()}}}
\label{\detokenize{mods:emu-pause}}\label{\detokenize{mods:pause}}
\sphinxAtStartPar
Pauses emulation.


\subsubsection{\sphinxstyleliteralintitle{\sphinxupquote{emu.unpause()}}}
\label{\detokenize{mods:emu-unpause}}\label{\detokenize{mods:unpause}}
\sphinxAtStartPar
Unpauses emulation.


\subsubsection{\sphinxstyleliteralintitle{\sphinxupquote{emu.emulateframe()}}}
\label{\detokenize{mods:emu-emulateframe}}\label{\detokenize{mods:emulateframe}}
\sphinxAtStartPar
Steps one frame. Equivalent to:

\begin{sphinxVerbatim}[commandchars=\\\{\}]
\PYG{n}{emu}\PYG{p}{.}\PYG{n}{setspeed}\PYG{p}{(}\PYG{l+s+s2}{\PYGZdq{}}\PYG{l+s+s2}{normal}\PYG{l+s+s2}{\PYGZdq{}}\PYG{p}{)}
\PYG{n}{emu}\PYG{p}{.}\PYG{n}{frameadvance}\PYG{p}{(}\PYG{p}{)}
\end{sphinxVerbatim}


\subsubsection{\sphinxstyleliteralintitle{\sphinxupquote{emu.emulateframefastnoskipping()}}}
\label{\detokenize{mods:emu-emulateframefastnoskipping}}\label{\detokenize{mods:emulateframefastnoskipping}}
\sphinxAtStartPar
Fast\sphinxhyphen{}forwards emulation once but render every frame. Equivalent to:

\begin{sphinxVerbatim}[commandchars=\\\{\}]
\PYG{n}{emu}\PYG{p}{.}\PYG{n}{setspeed}\PYG{p}{(}\PYG{l+s+s2}{\PYGZdq{}}\PYG{l+s+s2}{nothrottle}\PYG{l+s+s2}{\PYGZdq{}}\PYG{p}{)}
\PYG{n}{emu}\PYG{p}{.}\PYG{n}{frameadvance}\PYG{p}{(}\PYG{p}{)}
\end{sphinxVerbatim}


\subsubsection{\sphinxstyleliteralintitle{\sphinxupquote{emu.emulateframefast()}}}
\label{\detokenize{mods:emu-emulateframefast}}\label{\detokenize{mods:emulateframefast}}
\sphinxAtStartPar
Fast\sphinxhyphen{}forwards emulation once. Equivalent to:

\begin{sphinxVerbatim}[commandchars=\\\{\}]
\PYG{n}{emu}\PYG{p}{.}\PYG{n}{setspeed}\PYG{p}{(}\PYG{l+s+s2}{\PYGZdq{}}\PYG{l+s+s2}{turbo}\PYG{l+s+s2}{\PYGZdq{}}\PYG{p}{)}
\PYG{n}{emu}\PYG{p}{.}\PYG{n}{frameadvance}\PYG{p}{(}\PYG{p}{)}
\end{sphinxVerbatim}


\subsubsection{\sphinxstyleliteralintitle{\sphinxupquote{emu.emulateframeinvisible()}}}
\label{\detokenize{mods:emu-emulateframeinvisible}}\label{\detokenize{mods:emulateframeinvisible}}
\sphinxAtStartPar
Extremely\sphinxhyphen{}fast\sphinxhyphen{}forwards emulation once. Equivalent to:

\begin{sphinxVerbatim}[commandchars=\\\{\}]
\PYG{n}{emu}\PYG{p}{.}\PYG{n}{setspeed}\PYG{p}{(}\PYG{l+s+s2}{\PYGZdq{}}\PYG{l+s+s2}{maximum}\PYG{l+s+s2}{\PYGZdq{}}\PYG{p}{)}
\PYG{n}{emu}\PYG{p}{.}\PYG{n}{frameadvance}\PYG{p}{(}\PYG{p}{)}
\end{sphinxVerbatim}


\subsubsection{\sphinxstyleliteralintitle{\sphinxupquote{emu.redraw()}}}
\label{\detokenize{mods:emu-redraw}}\label{\detokenize{mods:redraw}}
\sphinxAtStartPar
Redraws the current frame.


\subsubsection{\sphinxstyleliteralintitle{\sphinxupquote{emu.getframecount()}}}
\label{\detokenize{mods:emu-getframecount}}\label{\detokenize{mods:getframecount}}
\sphinxAtStartPar
Gets the current frame count as an integer.


\subsubsection{\sphinxstyleliteralintitle{\sphinxupquote{emu.getlagcount()}}}
\label{\detokenize{mods:emu-getlagcount}}\label{\detokenize{mods:getlagcount}}
\sphinxAtStartPar
Gets the current total lag frames as an integer.


\subsubsection{\sphinxstyleliteralintitle{\sphinxupquote{emu.lagged()}}}
\label{\detokenize{mods:emu-lagged}}\label{\detokenize{mods:lagged}}
\sphinxAtStartPar
Returns true if the current frame is a lag frame.


\subsubsection{\sphinxstyleliteralintitle{\sphinxupquote{emu.emulating()}}}
\label{\detokenize{mods:emu-emulating}}\label{\detokenize{mods:emulating}}
\sphinxAtStartPar
Returns true if the emulator is currently running.


\subsubsection{\sphinxstyleliteralintitle{\sphinxupquote{emu.atframeboundary()}}}
\label{\detokenize{mods:emu-atframeboundary}}\label{\detokenize{mods:atframeboundary}}
\sphinxAtStartPar
Returns true if the emulator is at a frame boundary.


\subsubsection{\sphinxstyleliteralintitle{\sphinxupquote{emu.registerbefore(func)}}}
\label{\detokenize{mods:emu-registerbefore-func}}\label{\detokenize{mods:registerbefore}}
\sphinxAtStartPar
\sphinxcode{\sphinxupquote{func}} is a function taking no arguments.

\sphinxAtStartPar
Sets up a callback function to run before every frame starts. Can be used to e.g. set up memory or controller inputs for the next frame.

\sphinxAtStartPar
Example:

\begin{sphinxVerbatim}[commandchars=\\\{\}]
\PYG{k+kr}{function} \PYG{n+nf}{myCallback}\PYG{p}{(}\PYG{p}{)}
    \PYG{n+nb}{print}\PYG{p}{(}\PYG{l+s+s2}{\PYGZdq{}}\PYG{l+s+s2}{Frame starting}\PYG{l+s+s2}{\PYGZdq{}}\PYG{p}{)}
\PYG{k+kr}{end}

\PYG{n}{emu}\PYG{p}{.}\PYG{n}{registerbefore}\PYG{p}{(}\PYG{n}{myCallback}\PYG{p}{)}
\end{sphinxVerbatim}


\subsubsection{\sphinxstyleliteralintitle{\sphinxupquote{emu.registerafter(func)}}}
\label{\detokenize{mods:emu-registerafter-func}}\label{\detokenize{mods:registerafter}}
\sphinxAtStartPar
\sphinxcode{\sphinxupquote{func}} is a function taking no arguments.

\sphinxAtStartPar
Sets up a callback function to run after every frame ends. Can be used to e.g. read out memory or controller dataa for the previous frame.

\sphinxAtStartPar
Example:

\begin{sphinxVerbatim}[commandchars=\\\{\}]
\PYG{k+kr}{function} \PYG{n+nf}{myCallback}\PYG{p}{(}\PYG{p}{)}
    \PYG{n+nb}{print}\PYG{p}{(}\PYG{l+s+s2}{\PYGZdq{}}\PYG{l+s+s2}{Frame ending}\PYG{l+s+s2}{\PYGZdq{}}\PYG{p}{)}
\PYG{k+kr}{end}

\PYG{n}{emu}\PYG{p}{.}\PYG{n}{registerafter}\PYG{p}{(}\PYG{n}{myCallback}\PYG{p}{)}
\end{sphinxVerbatim}


\subsubsection{\sphinxstyleliteralintitle{\sphinxupquote{emu.registerstart(func)}}}
\label{\detokenize{mods:emu-registerstart-func}}\label{\detokenize{mods:registerstart}}
\sphinxAtStartPar
\sphinxcode{\sphinxupquote{func}} is a function taking no arguments.

\sphinxAtStartPar
Sets up a callback function to run on script start or reset. Does not work correctly with script start.

\sphinxAtStartPar
Example:

\begin{sphinxVerbatim}[commandchars=\\\{\}]
\PYG{k+kr}{function} \PYG{n+nf}{myCallback}\PYG{p}{(}\PYG{p}{)}
    \PYG{n+nb}{print}\PYG{p}{(}\PYG{n}{emu}\PYG{p}{.}\PYG{n}{gamecode}\PYG{p}{(}\PYG{p}{)}\PYG{p}{)}
    \PYG{n}{file} \PYG{o}{=} \PYG{n+nb}{io.open}\PYG{p}{(}\PYG{l+s+s2}{\PYGZdq{}}\PYG{l+s+s2}{log.txt}\PYG{l+s+s2}{\PYGZdq{}}\PYG{p}{,} \PYG{l+s+s2}{\PYGZdq{}}\PYG{l+s+s2}{w+}\PYG{l+s+s2}{\PYGZdq{}}\PYG{p}{)}
\PYG{k+kr}{end}

\PYG{n}{emu}\PYG{p}{.}\PYG{n}{registerstart}\PYG{p}{(}\PYG{n}{myCallback}\PYG{p}{)}
\end{sphinxVerbatim}


\subsubsection{\sphinxstyleliteralintitle{\sphinxupquote{emu.registerexit(func)}}}
\label{\detokenize{mods:emu-registerexit-func}}\label{\detokenize{mods:registerexit}}
\sphinxAtStartPar
\sphinxcode{\sphinxupquote{func}} is a function taking no arguments.

\sphinxAtStartPar
Sets up a callback function to run on script end.

\sphinxAtStartPar
Example:

\begin{sphinxVerbatim}[commandchars=\\\{\}]
\PYG{k+kr}{function} \PYG{n+nf}{myCallback}\PYG{p}{(}\PYG{p}{)}
    \PYG{n}{file}\PYG{p}{:}\PYG{n}{close}\PYG{p}{(}\PYG{p}{)}
\PYG{k+kr}{end}

\PYG{n}{emu}\PYG{p}{.}\PYG{n}{registerexit}\PYG{p}{(}\PYG{n}{myCallback}\PYG{p}{)}
\end{sphinxVerbatim}


\subsubsection{\sphinxstyleliteralintitle{\sphinxupquote{emu.persistglobalvariables(variabletable)}}}
\label{\detokenize{mods:emu-persistglobalvariables-variabletable}}\label{\detokenize{mods:persistglobalvariables}}
\sphinxAtStartPar
\sphinxcode{\sphinxupquote{variabletable}} is a table comprising keys and values.

\sphinxAtStartPar
Defines a global variable for each key in \sphinxcode{\sphinxupquote{variabletable}}. The variables will be set to the values they had the last time the script exited, or the value provided in the table if that is not available.
To set a default value of \sphinxcode{\sphinxupquote{nil}} for a variable, pass the variable name as a string with no associated value.

\sphinxAtStartPar
Example:

\begin{sphinxVerbatim}[commandchars=\\\{\}]
\PYG{n}{emu}\PYG{p}{.}\PYG{n}{persistglobalvariables}\PYG{p}{(}\PYG{p}{\PYGZob{}}
    \PYG{n}{variable1} \PYG{o}{=} \PYG{n}{defaultvalue1}\PYG{p}{,}
    \PYG{n}{variable2} \PYG{o}{=} \PYG{n}{defaultvalue2}\PYG{p}{,}
    \PYG{c+c1}{\PYGZhy{}\PYGZhy{} ...}
\PYG{p}{\PYGZcb{}}\PYG{p}{)}
\end{sphinxVerbatim}


\subsubsection{\sphinxstyleliteralintitle{\sphinxupquote{emu.message(str)}}}
\label{\detokenize{mods:emu-message-str}}\label{\detokenize{mods:message}}
\sphinxAtStartPar
\sphinxcode{\sphinxupquote{str}} is a string with the message to be displayed.

\sphinxAtStartPar
Displays an info message to the user.


\subsubsection{\sphinxstyleliteralintitle{\sphinxupquote{emu.print(...)}}}
\label{\detokenize{mods:emu-print}}\label{\detokenize{mods:print}}
\sphinxAtStartPar
Replacement for \sphinxcode{\sphinxupquote{luaB\_print()}} that outputs to the appropriate textbox instead of stdout.


\subsubsection{\sphinxstyleliteralintitle{\sphinxupquote{emu.openscript(filename)}}}
\label{\detokenize{mods:emu-openscript-filename}}\label{\detokenize{mods:openscript}}
\sphinxAtStartPar
\sphinxcode{\sphinxupquote{filename}} is a string containing the filepath of the script to open.

\sphinxAtStartPar
Opens a new Lua script. Only available in the Windows frontend.


\subsubsection{\sphinxstyleliteralintitle{\sphinxupquote{emu.reset()}}}
\label{\detokenize{mods:emu-reset}}\label{\detokenize{mods:reset}}
\sphinxAtStartPar
Resets the currently loaded ROM.


\subsubsection{\sphinxstyleliteralintitle{\sphinxupquote{emu.addmenu(menuName, menuEntries)}}}
\label{\detokenize{mods:emu-addmenu-menuname-menuentries}}\label{\detokenize{mods:addmenu}}
\sphinxAtStartPar
\sphinxcode{\sphinxupquote{menuName}} is a string containing the name the new menu should have. \sphinxcode{\sphinxupquote{menuEntries}} is a table containing the entries that the menu will have.

\sphinxAtStartPar
TODO: add an example for this.


\subsubsection{\sphinxstyleliteralintitle{\sphinxupquote{emu.setmenuiteminfo(menuItem, infoTable)}}}
\label{\detokenize{mods:emu-setmenuiteminfo-menuitem-infotable}}\label{\detokenize{mods:setmenuiteminfo}}
\sphinxAtStartPar
\sphinxcode{\sphinxupquote{menuItem}} is a… actually, I really have no idea.

\sphinxAtStartPar
TODO: work out how the hell this function works


\subsubsection{\sphinxstyleliteralintitle{\sphinxupquote{emu.registermenustart(func)}}}
\label{\detokenize{mods:emu-registermenustart-func}}\label{\detokenize{mods:registermenustart}}
\sphinxAtStartPar
\sphinxcode{\sphinxupquote{func}} is a function taking ?? arguments.

\sphinxAtStartPar
TODO: literally all of the menu stuff


\subsubsection{\sphinxstyleliteralintitle{\sphinxupquote{emu.register3devent(func)}}}
\label{\detokenize{mods:emu-register3devent-func}}\label{\detokenize{mods:register3devent}}
\sphinxAtStartPar
\sphinxcode{\sphinxupquote{func}} is a function taking ?? arguments.

\sphinxAtStartPar
TODO: figure out 3D stuff


\subsubsection{\sphinxstyleliteralintitle{\sphinxupquote{emu.set3dtransform(mode, matrix)}}}
\label{\detokenize{mods:emu-set3dtransform-mode-matrix}}\label{\detokenize{mods:set3dtransform}}
\sphinxAtStartPar
\sphinxcode{\sphinxupquote{mode}} is a number with a value of either 2 or 3. \sphinxcode{\sphinxupquote{matrix}} is a camera matrix.

\sphinxAtStartPar
TODO: 3D


\subsection{gui}
\label{\detokenize{mods:gui}}\label{\detokenize{mods:id2}}

\subsection{emu}
\label{\detokenize{mods:stylus}}\label{\detokenize{mods:id3}}

\subsection{savestate}
\label{\detokenize{mods:savestate}}\label{\detokenize{mods:id4}}

\subsection{memory}
\label{\detokenize{mods:memory}}\label{\detokenize{mods:id5}}
\sphinxAtStartPar
The \sphinxcode{\sphinxupquote{memory}} module contains functions related to game memory.
\begin{itemize}
\item {} 
\sphinxAtStartPar
{\hyperref[\detokenize{mods:readbyte}]{\sphinxcrossref{\DUrole{std,std-ref}{readbyte}}}}

\item {} 
\sphinxAtStartPar
{\hyperref[\detokenize{mods:readbytesigned}]{\sphinxcrossref{\DUrole{std,std-ref}{readbytesigned}}}}

\item {} 
\sphinxAtStartPar
{\hyperref[\detokenize{mods:readword}]{\sphinxcrossref{\DUrole{std,std-ref}{readword}}}}

\item {} 
\sphinxAtStartPar
{\hyperref[\detokenize{mods:readwordsigned}]{\sphinxcrossref{\DUrole{std,std-ref}{readwordsigned}}}}

\item {} 
\sphinxAtStartPar
{\hyperref[\detokenize{mods:readdword}]{\sphinxcrossref{\DUrole{std,std-ref}{readdword}}}}

\item {} 
\sphinxAtStartPar
{\hyperref[\detokenize{mods:readdwordsigned}]{\sphinxcrossref{\DUrole{std,std-ref}{readdwordsigned}}}}

\item {} 
\sphinxAtStartPar
{\hyperref[\detokenize{mods:readbyterange}]{\sphinxcrossref{\DUrole{std,std-ref}{readbyterange}}}}

\item {} 
\sphinxAtStartPar
{\hyperref[\detokenize{mods:writebyte}]{\sphinxcrossref{\DUrole{std,std-ref}{writebyte}}}}

\item {} 
\sphinxAtStartPar
{\hyperref[\detokenize{mods:writeword}]{\sphinxcrossref{\DUrole{std,std-ref}{writeword}}}}

\item {} 
\sphinxAtStartPar
{\hyperref[\detokenize{mods:writedword}]{\sphinxcrossref{\DUrole{std,std-ref}{writedword}}}}

\item {} 
\sphinxAtStartPar
{\hyperref[\detokenize{mods:isvalid}]{\sphinxcrossref{\DUrole{std,std-ref}{isvalid}}}}

\item {} 
\sphinxAtStartPar
{\hyperref[\detokenize{mods:getregister}]{\sphinxcrossref{\DUrole{std,std-ref}{getregister}}}}

\item {} 
\sphinxAtStartPar
{\hyperref[\detokenize{mods:setregister}]{\sphinxcrossref{\DUrole{std,std-ref}{setregister}}}}

\item {} 
\sphinxAtStartPar
{\hyperref[\detokenize{mods:vram-readword}]{\sphinxcrossref{\DUrole{std,std-ref}{vram\_readword}}}}

\item {} 
\sphinxAtStartPar
{\hyperref[\detokenize{mods:vram-writeword}]{\sphinxcrossref{\DUrole{std,std-ref}{vram\_writeword}}}}

\item {} 
\sphinxAtStartPar
{\hyperref[\detokenize{mods:registerwrite}]{\sphinxcrossref{\DUrole{std,std-ref}{registerwrite}}}}

\item {} 
\sphinxAtStartPar
{\hyperref[\detokenize{mods:registerread}]{\sphinxcrossref{\DUrole{std,std-ref}{registerread}}}}

\item {} 
\sphinxAtStartPar
{\hyperref[\detokenize{mods:registerexec}]{\sphinxcrossref{\DUrole{std,std-ref}{registerexec}}}}

\end{itemize}


\subsubsection{\sphinxstyleliteralintitle{\sphinxupquote{memory.readbyte(address)}}}
\label{\detokenize{mods:memory-readbyte-address}}\label{\detokenize{mods:readbyte}}
\sphinxAtStartPar
\sphinxcode{\sphinxupquote{address}} is an integer representing the address of a byte in the ARM9 CPU’s address space.

\sphinxAtStartPar
Reads one byte from address \sphinxcode{\sphinxupquote{address}}.


\subsubsection{\sphinxstyleliteralintitle{\sphinxupquote{memory.readbytesigned(address)}}}
\label{\detokenize{mods:memory-readbytesigned-address}}\label{\detokenize{mods:readbytesigned}}
\sphinxAtStartPar
\sphinxcode{\sphinxupquote{address}} is an integer representing the address of a signed byte in the ARM9 CPU’s address space.

\sphinxAtStartPar
Reads one signed byte from address \sphinxcode{\sphinxupquote{address}}.


\subsubsection{\sphinxstyleliteralintitle{\sphinxupquote{memory.readword(address)}}}
\label{\detokenize{mods:memory-readword-address}}\label{\detokenize{mods:readword}}
\sphinxAtStartPar
\sphinxcode{\sphinxupquote{address}} is an integer representing the address of a word in the ARM9 CPU’s address space.

\sphinxAtStartPar
Reads one word (16 bits) from address \sphinxcode{\sphinxupquote{address}}.


\subsubsection{\sphinxstyleliteralintitle{\sphinxupquote{memory.readwordsigned(address)}}}
\label{\detokenize{mods:memory-readwordsigned-address}}\label{\detokenize{mods:readwordsigned}}
\sphinxAtStartPar
\sphinxcode{\sphinxupquote{address}} is an integer representing the address of a signed word in the ARM9 CPU’s address space.

\sphinxAtStartPar
Reads one signed word (16 bits) from address \sphinxcode{\sphinxupquote{address}}.


\subsubsection{\sphinxstyleliteralintitle{\sphinxupquote{memory.readdword(address)}}}
\label{\detokenize{mods:memory-readdword-address}}\label{\detokenize{mods:readdword}}
\sphinxAtStartPar
\sphinxcode{\sphinxupquote{address}} is an integer representing the address of a dword in the ARM9 CPU’s address space.

\sphinxAtStartPar
Reads one dword (32 bits) from address \sphinxcode{\sphinxupquote{address}}.


\subsubsection{\sphinxstyleliteralintitle{\sphinxupquote{memory.readdwordsigned(address)}}}
\label{\detokenize{mods:memory-readdwordsigned-address}}\label{\detokenize{mods:readdwordsigned}}
\sphinxAtStartPar
\sphinxcode{\sphinxupquote{address}} is an integer representing the address of a signed dword in the ARM9 CPU’s address space.

\sphinxAtStartPar
Reads one signed dword (32 bits) from address \sphinxcode{\sphinxupquote{address}}.


\subsubsection{\sphinxstyleliteralintitle{\sphinxupquote{memory.readbyterange(address, length)}}}
\label{\detokenize{mods:memory-readbyterange-address-length}}\label{\detokenize{mods:readbyterange}}
\sphinxAtStartPar
\sphinxcode{\sphinxupquote{address}} is an integer representing the start of the block to read. \sphinxcode{\sphinxupquote{length}} is an integer representing the length of the block to read.

\sphinxAtStartPar
Reads \sphinxcode{\sphinxupquote{length}} bytes from address \sphinxcode{\sphinxupquote{address}}. Returns an array.


\subsubsection{\sphinxstyleliteralintitle{\sphinxupquote{memory.writebyte(address, value)}}}
\label{\detokenize{mods:memory-writebyte-address-value}}\label{\detokenize{mods:writebyte}}
\sphinxAtStartPar
\sphinxcode{\sphinxupquote{address}} is an integer representing the address of a byte in the ARM9 CPU’s address space. \sphinxcode{\sphinxupquote{value}} is the byte to write there as an integer.

\sphinxAtStartPar
Writes the byte \sphinxcode{\sphinxupquote{value}} to address \sphinxcode{\sphinxupquote{address}}.


\subsubsection{\sphinxstyleliteralintitle{\sphinxupquote{memory.writeword(address, value)}}}
\label{\detokenize{mods:memory-writeword-address-value}}\label{\detokenize{mods:writeword}}
\sphinxAtStartPar
\sphinxcode{\sphinxupquote{address}} is an integer representing the address of a word in the ARM9 CPU’s address space. \sphinxcode{\sphinxupquote{value}} is the word to write there as an integer.

\sphinxAtStartPar
Writes the word \sphinxcode{\sphinxupquote{value}} to address \sphinxcode{\sphinxupquote{address}}.


\subsubsection{\sphinxstyleliteralintitle{\sphinxupquote{memory.writedword(address, value)}}}
\label{\detokenize{mods:memory-writedword-address-value}}\label{\detokenize{mods:writedword}}
\sphinxAtStartPar
\sphinxcode{\sphinxupquote{address}} is an integer representing the address of a dword in the ARM9 CPU’s address space. \sphinxcode{\sphinxupquote{value}} is the dword to write there as an integer.

\sphinxAtStartPar
Writes the dword \sphinxcode{\sphinxupquote{value}} to address \sphinxcode{\sphinxupquote{address}}.


\subsubsection{\sphinxstyleliteralintitle{\sphinxupquote{memory.isvalid(address)}}}
\label{\detokenize{mods:memory-isvalid-address}}\label{\detokenize{mods:isvalid}}
\sphinxAtStartPar
\sphinxcode{\sphinxupquote{address}} is an integer representing an address in the ARM9 CPU’s address space.

\sphinxAtStartPar
Returns \sphinxcode{\sphinxupquote{true}} if \sphinxcode{\sphinxupquote{address}} is a valid hardware address, else \sphinxcode{\sphinxupquote{false}}.

\sphinxAtStartPar
Example:

\begin{sphinxVerbatim}[commandchars=\\\{\}]
\PYG{k+kr}{function} \PYG{n+nf}{getString}\PYG{p}{(}\PYG{n}{address}\PYG{p}{)}
    \PYG{k+kr}{if} \PYG{n}{memory}\PYG{p}{.}\PYG{n}{isvalid}\PYG{p}{(}\PYG{n}{address}\PYG{p}{)} \PYG{k+kr}{then}
        \PYG{k+kd}{local} \PYG{n}{index} \PYG{o}{=} \PYG{l+m+mi}{1}
        \PYG{k+kd}{local} \PYG{n}{str} \PYG{o}{=} \PYG{l+s+s2}{\PYGZdq{}}\PYG{l+s+s2}{\PYGZdq{}}
        \PYG{k+kd}{local} \PYG{n}{c} \PYG{o}{=} \PYG{n}{memory}\PYG{p}{.}\PYG{n}{readbyte}\PYG{p}{(}\PYG{n}{address}\PYG{p}{)}
        \PYG{k+kr}{while} \PYG{n}{c} \PYG{o}{\PYGZti{}=} \PYG{l+m+mi}{0} \PYG{k+kr}{do}
            \PYG{n}{str} \PYG{o}{=} \PYG{n}{str} \PYG{o}{..} \PYG{n+nb}{string.char}\PYG{p}{(}\PYG{n}{c}\PYG{p}{)}
            \PYG{n}{c} \PYG{o}{=} \PYG{n}{memory}\PYG{p}{.}\PYG{n}{readbyte}\PYG{p}{(}\PYG{n}{address} \PYG{o}{+} \PYG{n}{index}\PYG{p}{)}
            \PYG{n}{index} \PYG{o}{=} \PYG{n}{index} \PYG{o}{+} \PYG{l+m+mi}{1}
        \PYG{k+kr}{end}
        \PYG{k+kr}{return} \PYG{n}{str}
    \PYG{k+kr}{end}
    \PYG{k+kr}{return} \PYG{k+kc}{nil}
\PYG{k+kr}{end}
\end{sphinxVerbatim}


\subsubsection{\sphinxstyleliteralintitle{\sphinxupquote{memory.getregister(cpu\_dot\_registername\_string)}}}
\label{\detokenize{mods:memory-getregister-cpu-dot-registername-string}}\label{\detokenize{mods:getregister}}
\sphinxAtStartPar
\sphinxcode{\sphinxupquote{cpu\_dot\_registername\_string}} is a string representing the register to read. The format is \sphinxcode{\sphinxupquote{\textless{}CPU\textgreater{}.\textless{}register\textgreater{}}}, where \sphinxcode{\sphinxupquote{CPU}} is “arm9” or “arm7” (or “main” or “sub”, respectively) and \sphinxcode{\sphinxupquote{register}} is \sphinxcode{\sphinxupquote{r0}}\sphinxhyphen{}\sphinxcode{\sphinxupquote{r15}}, \sphinxcode{\sphinxupquote{cpsr}} or \sphinxcode{\sphinxupquote{spsr}}.

\sphinxAtStartPar
Returns the contents of the register referenced by \sphinxcode{\sphinxupquote{cpu\_dot\_registername\_string}}.

\sphinxAtStartPar
Example:

\begin{sphinxVerbatim}[commandchars=\\\{\}]
\PYG{k+kr}{function} \PYG{n+nf}{debugPrintHook}\PYG{p}{(}\PYG{p}{)}
    \PYG{k+kd}{local} \PYG{n}{strAddr} \PYG{o}{=} \PYG{n}{memory}\PYG{p}{.}\PYG{n}{getregister}\PYG{p}{(}\PYG{l+s+s2}{\PYGZdq{}}\PYG{l+s+s2}{arm9.r0}\PYG{l+s+s2}{\PYGZdq{}}\PYG{p}{)}
    \PYG{k+kr}{if} \PYG{n}{strAddr} \PYG{o}{\PYGZti{}=} \PYG{l+m+mi}{0} \PYG{k+kr}{then}
        \PYG{k+kd}{local} \PYG{n}{str} \PYG{o}{=} \PYG{n}{getString}\PYG{p}{(}\PYG{n}{strAddr}\PYG{p}{)}
        \PYG{k+kr}{if} \PYG{n}{str} \PYG{o}{\PYGZti{}=} \PYG{k+kc}{nil} \PYG{k+kr}{then}
            \PYG{n+nb}{print}\PYG{p}{(}\PYG{n}{str}\PYG{p}{)}
        \PYG{k+kr}{end}
    \PYG{k+kr}{end}
\PYG{k+kr}{end}
\end{sphinxVerbatim}


\subsubsection{\sphinxstyleliteralintitle{\sphinxupquote{memory.setregister(cpu\_dot\_registername\_string, value)}}}
\label{\detokenize{mods:memory-setregister-cpu-dot-registername-string-value}}\label{\detokenize{mods:setregister}}
\sphinxAtStartPar
\sphinxcode{\sphinxupquote{cpu\_dot\_registername\_string}} is a string representing the register to write. The format is \sphinxcode{\sphinxupquote{\textless{}CPU\textgreater{}.\textless{}register\textgreater{}}}, where \sphinxcode{\sphinxupquote{CPU}} is “arm9” or “arm7” (or “main” or “sub”, respectively) and \sphinxcode{\sphinxupquote{register}} is \sphinxcode{\sphinxupquote{r0}}\sphinxhyphen{}\sphinxcode{\sphinxupquote{r15}}, \sphinxcode{\sphinxupquote{cpsr}} or \sphinxcode{\sphinxupquote{spsr}}. \sphinxcode{\sphinxupquote{value}} is the data to write to it, as an integer.

\sphinxAtStartPar
Writes \sphinxcode{\sphinxupquote{value}} to the register referenced by \sphinxcode{\sphinxupquote{cpu\_dot\_registername\_string}}.


\subsubsection{\sphinxstyleliteralintitle{\sphinxupquote{memory.vram\_readword(address)}}}
\label{\detokenize{mods:memory-vram-readword-address}}\label{\detokenize{mods:vram-readword}}
\sphinxAtStartPar
\sphinxcode{\sphinxupquote{address}} is the address of a word in the console’s VRAM address space.

\sphinxAtStartPar
Reads one word (16 bits) from VRAM address \sphinxcode{\sphinxupquote{address}}.


\subsubsection{\sphinxstyleliteralintitle{\sphinxupquote{memory.vram\_writeword(address, value)}}}
\label{\detokenize{mods:memory-vram-writeword-address-value}}\label{\detokenize{mods:vram-writeword}}
\sphinxAtStartPar
\sphinxcode{\sphinxupquote{address}} is the address of a word in the console’s VRAM address space. \sphinxcode{\sphinxupquote{value}} is the word to write there.

\sphinxAtStartPar
Writes one word (16 bits) to VRAM address \sphinxcode{\sphinxupquote{address}}.


\subsubsection{\sphinxstyleliteralintitle{\sphinxupquote{memory.registerwrite(address{[}, size = 1{]}{[}, cpuname = "main"{]}, func)}}}
\label{\detokenize{mods:memory-registerwrite-address-size-1-cpuname-main-func}}\label{\detokenize{mods:registerwrite}}
\sphinxAtStartPar
\sphinxcode{\sphinxupquote{address}} is an integer representing an address in the chosen CPU’s address space. \sphinxcode{\sphinxupquote{size}} is an integer representing the length of the memory region that should be registered. \sphinxcode{\sphinxupquote{cpuname}} is a string containing the name of the CPU to register to, with a value of either “main” or “sub” (or “arm9” or “arm7”, respectively). \sphinxcode{\sphinxupquote{func}} is a function taking no arguments.

\sphinxAtStartPar
Registers a memory write watchpoint with size \sphinxcode{\sphinxupquote{size}} at address \sphinxcode{\sphinxupquote{cpuname}}:\sphinxcode{\sphinxupquote{address}}. When \sphinxcode{\sphinxupquote{cpuname}}:\sphinxcode{\sphinxupquote{address}} is written, \sphinxcode{\sphinxupquote{func}} will be called.

\sphinxAtStartPar
TODO: verify whether \sphinxcode{\sphinxupquote{size}} and \sphinxcode{\sphinxupquote{cpuname}} are used at all (they don’t seem to be).


\subsubsection{\sphinxstyleliteralintitle{\sphinxupquote{memory.registerread(address{[}, size = 1{]}{[}, cpuname = "main"{]}, func)}}}
\label{\detokenize{mods:memory-registerread-address-size-1-cpuname-main-func}}\label{\detokenize{mods:registerread}}
\sphinxAtStartPar
\sphinxcode{\sphinxupquote{address}} is an integer representing an address in the chosen CPU’s address space. \sphinxcode{\sphinxupquote{size}} is an integer representing the length of the memory region that should be registered. \sphinxcode{\sphinxupquote{cpuname}} is a string containing the name of the CPU to register to, with a value of either “main” or “sub” (or “arm9” or “arm7”, respectively). \sphinxcode{\sphinxupquote{func}} is a function taking no arguments.

\sphinxAtStartPar
Registers a memory read watchpoint with size \sphinxcode{\sphinxupquote{size}} at address \sphinxcode{\sphinxupquote{cpuname}}:\sphinxcode{\sphinxupquote{address}}. When \sphinxcode{\sphinxupquote{cpuname}}:\sphinxcode{\sphinxupquote{address}} is read, \sphinxcode{\sphinxupquote{func}} will be called.

\sphinxAtStartPar
TODO: verify whether \sphinxcode{\sphinxupquote{size}} and \sphinxcode{\sphinxupquote{cpuname}} are used at all (they don’t seem to be).


\subsubsection{\sphinxstyleliteralintitle{\sphinxupquote{memory.registerexec(address{[}, size = 2{]}{[}, cpuname = "main"{]}, func)}}}
\label{\detokenize{mods:memory-registerexec-address-size-2-cpuname-main-func}}\label{\detokenize{mods:registerexec}}
\sphinxAtStartPar
\sphinxcode{\sphinxupquote{address}} is an integer representing the address of an instruction in the chosen CPU’s address space. \sphinxcode{\sphinxupquote{size}} is an integer representing the length of the memory region that should be registered. \sphinxcode{\sphinxupquote{cpuname}} is a string containing the name of the CPU to register to, with a value of either “main” or “sub” (or “arm9” or “arm7”, respectively). \sphinxcode{\sphinxupquote{func}} is a function taking no arguments.

\sphinxAtStartPar
Registers a memory exec watchpoint with size \sphinxcode{\sphinxupquote{size}} at address \sphinxcode{\sphinxupquote{cpuname}}:\sphinxcode{\sphinxupquote{address}}. When \sphinxcode{\sphinxupquote{cpuname}}:\sphinxcode{\sphinxupquote{address}} is executed, \sphinxcode{\sphinxupquote{func}} will be called.

\sphinxAtStartPar
TODO: verify whether \sphinxcode{\sphinxupquote{size}} and \sphinxcode{\sphinxupquote{cpuname}} are used at all (they don’t seem to be).

\sphinxAtStartPar
Example:

\begin{sphinxVerbatim}[commandchars=\\\{\}]
\PYG{n}{memory}\PYG{p}{.}\PYG{n}{registerexec}\PYG{p}{(}\PYG{l+m+mh}{0x02103ea8}\PYG{p}{,} \PYG{n}{debugPrintHook}\PYG{p}{)}
\end{sphinxVerbatim}


\subsection{joypad}
\label{\detokenize{mods:joypad}}\label{\detokenize{mods:id6}}

\subsection{input}
\label{\detokenize{mods:input}}\label{\detokenize{mods:id7}}

\subsection{movie}
\label{\detokenize{mods:movie}}\label{\detokenize{mods:id8}}

\subsection{sound}
\label{\detokenize{mods:sound}}\label{\detokenize{mods:id9}}

\subsection{bit}
\label{\detokenize{mods:bit}}\label{\detokenize{mods:id10}}

\subsection{agg}
\label{\detokenize{mods:agg}}\label{\detokenize{mods:id11}}

\subsection{controller}
\label{\detokenize{mods:controller}}\label{\detokenize{mods:id12}}

\chapter{Indices and tables}
\label{\detokenize{index:indices-and-tables}}\begin{itemize}
\item {} 
\sphinxAtStartPar
\DUrole{xref,std,std-ref}{modindex}

\end{itemize}



\renewcommand{\indexname}{Index}
\printindex
\end{document}